\documentclass{beamer}
\usepackage[utf8]{inputenc}
\usepackage[brazil]{babel}
\usetheme{Warsaw}

\title{Analisando Composições de Serviços com Apache Hadoop}
\subtitle{Coreografia \emph{versus} Orquestração}
\author{Carlos Eduardo M. Santos, Thiago Furtado}\institute{Universidade de São Paulo - IME\\MAC0431 - Introdução à Computação Paralela e Distribuída}
\date{19 de outubro de 2012}

\begin{document}

\begin{frame}
\titlepage
\end{frame}

\section{Introdução}

\begin{frame}
É fácil analisar o tempo de resposta do serviço chamado diretamente.
\begin{figure}
\includegraphics[width=\linewidth,clip=true,trim=1mm 2mm 3mm 20mm]{figures/portals1-4}
\caption{Chamadas a operações do serviço \emph{Portal}.}
\end{figure}
\end{frame}

\begin{frame}
Mas e dos outros serviços participantes da composição?
\begin{figure}
\includegraphics[width=\textwidth,clip=true,trim=7mm 9mm 8mm 16mm]{figures/orch}
\caption{Interação entre serviços na versão orquestrada.}
\end{figure}
\end{frame}

\begin{frame}
\begin{figure}
\includegraphics[height=0.85\textheight,clip=true,trim=2mm 10mm 5mm 14mm]{figures/chor}
\caption{Interação entre serviços na versão coreografada.}
\end{figure}
\end{frame}

\begin{frame}
\begin{itemize}
  \item Gerar carga em um serviço por vez?
  \begin{itemize}
    \item Dependências entre serviços;
    \item Muitos estados, entradas e saídas possíveis;
    \item Outros serviços numa mesma máquina;
  \end{itemize}
  \item Analizar o log dos servidores:
  \begin{itemize}
    \item Sem necessidade de realizar outros testes;
    \item Permite comparar com o resultado de toda a composição;
    \item Problemas? Logs: tamanho, formato.
  \end{itemize}
\end{itemize}
\end{frame}

\section{Dados}
\begin{frame}
\end{frame}

\end{document}
