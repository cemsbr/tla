\documentclass[brazil, a4paper,12pt]{article}
\usepackage[brazil]{babel}
\usepackage{graphicx}
\usepackage{geometry}
\usepackage[utf8]{inputenc}
\usepackage[T1]{fontenc}
\usepackage{url}
\usepackage{hyperref}
\usepackage{indentfirst}
\usepackage[usenames]{color}
\geometry{a4paper,left=3cm,right=3cm,top=2.5cm,bottom=2.5cm}

\begin{document}
\begin{titlepage}

  \vfill

  \begin{center}
    \begin{large}
      Universidade de São Paulo
    \end{large}
  \end{center}

  \begin{center}
    \begin{large}
      Instituto de Matemática e Estatística
    \end{large}
  \end{center}

  \begin{center}
    \begin{large}
      Programa de Pós-Graduação em Ciência da Computação
    \end{large}
  \end{center}

  \vfill

  \begin{center}
    \begin{Large}
        \textbf{MAC0431}\\
        \textbf{Introdução à Computação Paralela e Distribuída}\\
          Segundo Exercício Programa\\
    \end{Large}
  \end{center}


  \vfill

  \begin{center}
    \begin{large}
      Carlos Eduardo Moreira dos Santos\\
      Thiago Furtado de Mendonça
    \end{large}
  \end{center}

  \begin{center}
    \begin{large}
      Professor - Alfredo Goldman\\
    \end{large}
  \end{center}

  \vfill

  \begin{center}
    \begin{large}
      São Paulo \\
      \today \\
    \end{large}
  \end{center}

\clearpage
\tableofcontents 
\end{titlepage}

\section{Introdução}

Este trabalho tem o objetivo

\subsection{Caso de Uso}

Future market

orch
chor

resulta do portal

\section{Dados}

\subsection{Gerador de Carga}
Saída do gerador de carga (125 MB):

\begin{verbatim}
# orch,1,50,1349926295888
424
892
(...)
2017
2038
# orch,1,100,1349926297945
1110
1189
(...)
\end{verbatim}

\begin{description}
  \item[orch,1] Arquitetura
  \item[50/100] Número de clientes simultâneos (eixo x)
  \item[134552629...] Tempo em milissegundos
\end{description}

\subsection{Log}
  Formato modificado do log de Apache Tomcat (200 GB):
\begin{verbatim}
198.55.32.149 - - [09/Oct/2012:02:47:56 -0700] "POST
/supermarket5/orchestration HTTP/1.1" 200 914 811
\end{verbatim}

    A qual teste esse acesso pertence?

    \section{Hadoop}
    \subsection{Map}
    Pré-requisito: pela saída do gerador de carga, gerar uma tabela de horários e   parâmetros dos testes. Utilizando-a, podemos obter \emph{orch,1,50} através de  \emph{10/Oct/2012:20:31:35}.

\subsection{Reduce}
Para cada conjunto de parâmetros (arquitetura, teste), as médias e seus         intervalos de confiança serão mostrados em gráficos.

\begin{verbatim}
REDUCE
while has value
    values.store(value)
  mean <- get_mean(values)
  ic <- get_ic(values, mean)
  output(key, "mean,ic")
  \end{verbatim}

  \subsection{Cuidados}
  Alguns cuidados são necessários nos cálculos da redução:
  \begin{itemize}
  \item \emph{Overflow} numérico
  \item \emph{Overflow} de memória
  \end{itemize}

%\begin{figure}[!ht]
%  \begin{center}
%    \includegraphics[width=5cm]{fig001.jpg}
%  \end{center}
%  \caption{Estrutura da Pilha}
%  \label{fig:exemplo}
%\end{figure}
%
%Muito dos algoritmos são extraídos de~\cite{ziviani:2004}.
%
%\bibliographystyle{plain}
%\bibliography{modelo}

\end{document}
